\documentclass[12pt]{exam}

%Bring in the packages I'll need normally
\usepackage{amsmath} %AMS Math Package
\usepackage{amssymb} %Math symbols like \mathbb
\usepackage{multicol} % Allows for multiple columns
\usepackage{graphicx} %Allows images to be inserted using \includegraphics
\usepackage{enumitem} %Allows for fancier lists, use [noitemsep] or [noitemsep, nolistsep] after \begin{}
\usepackage{hyperref} %Allows the use of web links (\url, \href) and computer paths (\path)

\usepackage[version=3]{mhchem} %Simpler chemistry notation, use \ce{} to typeset chemical formula
	%e.g. \ce{H2O} for water and \ce{1/2SO4^2-} to set half a mol of sulfate ion.

%Set the page to be letter sized with 1" margins
\usepackage[dvips,letterpaper,margin=1in]{geometry}

% See https://www.sharelatex.com/learn/Typing_exams_in_LaTeX
\usepackage{tabularx}
\newcolumntype{Y}{>{\centering\arraybackslash}X}

\rhead{\textbf{Listening questionnaire}}
\lhead{\textbf{Sonification of \ce{NO2} trends}}
\headrule


\begin{document}
\printanswers % comment out to hide answers

%%%%%%%%%%%%%%%%%%%%%%%%
%%%%% Command defs %%%%%
%%%%%%%%%%%%%%%%%%%%%%%%

\renewcommand{\questionshook}{\setlength{\itemsep}{1cm}}

\newcommand{\fiveptincdec}{%
\begin{center}
\begin{tabularx}{\textwidth}{YYYYY}
$\square$ & $\square$ & $\square$ & $\square$ & $\square$ \\
Strongly decreasing & Weakly decreasing & No noticeable change & Weakly increasing & Strongly increasing
\end{tabularx}
\end{center}
}

\newcommand{\fiveptmag}{%
\begin{center}
\begin{tabularx}{\textwidth}{YYYYY}
$\square$ & $\square$ & $\square$ & $\square$ & $\square$ \\
Much less & Slightly less & No difference & Slightly greater & Much greater
\end{tabularx}
\end{center}
}

\newcommand{\fiveptcorrna}{%
\begin{center}
\begin{tabularx}{\textwidth}{YYYYYY}
$\square$ & $\square$ & $\square$ & $\square$ & $\square$ & $\square$ \\
Strongly anti-correlated & Weakly anti-correlated & No significant correlation & Weakly correlated & Strongly correlated & Could not tell
\end{tabularx}
\end{center}
}

\newcommand{\fiveptgeneric}[2]{%
\begin{center}
\begin{tabularx}{\textwidth}{YYYYY}
1 & 2 & 3 & 4 & 5 \\
$\square$ & $\square$ & $\square$ & $\square$ & $\square$ \\
#1 & & & & #2
\end{tabularx}
\end{center}
}

\newcommand{\fiveptgna}[4][]{%
\begin{center}
\begin{tabularx}{\textwidth}{YYYYYY}
1 & 2 & 3 & 4 & 5 & N/A \\
$\square$ & $\square$ & $\square$ & $\square$ & $\square$ & $\square$\\
#2 & & #1 & & #3 & #4
\end{tabularx}
\end{center}
}



\newcommand{\instructions}[1]{%
\vspace{1in}%
\begin{EnvUplevel}
#1
\end{EnvUplevel}
}

%%%%%%%%%%%%%%%%%%%%%%%%%%%%%%%
%%%%% Begin survey proper %%%%%
%%%%%%%%%%%%%%%%%%%%%%%%%%%%%%%

\section*{Understanding trends through sound}

\noindent This first section will guide you through exploring \ce{NO2} trends through the sonifcation interface. To begin, use the following settings:
\begin{itemize}[noitemsep]
\item All cities \textbf{ON}
\item Looping \textbf{ON} or \textbf{OFF}
\item Speed \textbf{20}
\item \ce{NO2} \textbf{ON}, \ce{O3} \textbf{OFF}
\item Season filter \textbf{OFF}
\item Synth sound based on location \textbf{ON}
\item Scaling \textbf{RELATIVE TO INDIVIDUAL MAXIMUM}
\item Panning \textbf{SITE LOCATION}
\end{itemize}

\noindent Listen to the data record as many times as necessary to answer the following questions.

\begin{questions}
\setcounter{question}{-1}
\question Please indicate whether you are listening on a stereo or 8-channel setup\\
	\begin{oneparchoices}
	\choice Stereo speakers
	\choice Stereo headphones
	\choice 8-channel
	\end{oneparchoices}


\question \ce{NO2} from cities:
	\begin{parts}
	\part Is the overall trend in cities' \ce{NO2} increasing, decreasing, or staying constant?
	\fiveptincdec
	
	\begin{solution}
	\emph{Strongly decreasing} best, \emph{weakly decreasing} acceptable
	\end{solution}

	\part Rate your confidence in this answer.
	\fiveptgeneric{Minimal confidence}{Extremely confident}

	\end{parts}

\newpage
\instructions{%
Now turn \textbf{OFF} all the cities and turn \textbf{ON} all the power plants. Leave the other settings the same.
}

\question \ce{NO2} from power plants:
	\begin{parts}
	\part Is the overall trend in power plants' \ce{NO2} increasing, decreasing, or staying constant?
	\fiveptincdec

	\begin{solution}
	\emph{Strongly decreasing} best, \emph{weakly decreasing} acceptable
	\end{solution}
	
	\part Rate your confidence in this answer.
	\fiveptgeneric{Minimal confidence}{Extremely confident}

	\end{parts}


\instructions{%
Now turn \textbf{OFF} all the power plants and turn \textbf{ON} all the rural sites. Leave the other settings the same.
}

\question
	\begin{parts}
	\part Is the overall trend in rural sites' \ce{NO2} increasing, decreasing, or staying constant?
	\fiveptincdec

	\begin{solution}
	\emph{Staying constant} is most correct, some may say \emph{weakly decreasing}
	\end{solution}
	
	\part Rate your confidence in this answer.
	\fiveptgeneric{Minimal confidence}{Extremely confident}
	\end{parts}


\instructions{%
Now \textbf{PAUSE} and \textbf{MUTE} the sound, then turn \textbf{OFF} all the rural sites and turn \textbf{ON} the following sites:
	\begin{itemize}
	\item Atlanta (city)
	\item Cholla (power plant)
	\item Crater Lake (rural)
	\end{itemize}
Decrease the speed to \textbf{10}. Turning \textbf{ON} looping will probably help.
}

\question Inter-category correlation:
	\begin{parts}
	\part How strongly are Atlanta and Cholla's \ce{NO2} concentrations correlated?
	\fiveptcorrna
	
	\begin{solution}
	Atlanta and Cholla are pretty strongly correlated (esp. for satellite data): $R^2 = 0.48$ (line of best fit = $0.21x + 1.5 \times 10^{14}$ for Cholla vs. Atlanta). \emph{Strongly correlated} is the best answer.
	\end{solution}
	
	\part Rate your confidence in this answer.
	\fiveptgeneric{Minimal confidence}{Extremely confident}
	
	\part How strongly are Atlanta and Crater Lake's \ce{NO2} concentrations correlated?
	\fiveptcorrna
	
	\begin{solution}
	Atlanta and Crater lake are not really correlated: $R^2 = 0.09$ (line of best fit = $-0.07x + 8.4 \times 10^{14}$ for Crater Lake vs. Atlanta). \emph{No significant correlation} is the best answer.
	\end{solution}
	
	\part Rate your confidence in this answer.
	\fiveptgeneric{Minimal confidence}{Extremely confident}
	\end{parts}


\instructions{%
Now turn \textbf{OFF} those three sites and turn \textbf{ON} the following cities:
	\begin{itemize} [noitemsep, nolistsep]
	\item Indianapolis, IN
	\item Los Angeles, VA
	\item Reno, NV
	\item San Antonio, TX
	\end{itemize}
Also turn \textbf{ON} the season filter and decrease the speed to \textbf{2--4}. 
}

\question Seasonal variation:
	\begin{parts}
	\part Is the \ce{NO2} in the summer greater or less than in the winter?
	\fiveptmag

	\begin{solution}
	The mean summer--winter difference is $-27.5\% \pm 26.5\%$ ($1\sigma$), the median difference is $-29.8\%$ with quartiles of $[-42.3\%, -19.5\%]$. Therefore, I'd say \emph{slightly less} is the best answer, but \emph{much less} is also valid.
	\end{solution}

	\part Rate your confidence in this answer.
	\fiveptgeneric{Minimal confidence}{Extremely confident}

	\end{parts}

\instructions{Turn the seasonal filter \textbf{OFF}. Reset the speed to \textbf{20}.}	

\question Spatialization:
	\begin{parts}
	\part Keep all four cities on. Which city has the largest relative change in \ce{NO2}? \\
		\begin{choices}
		\CorrectChoice Indianapolis, IN
		\CorrectChoice Los Angeles, CA
		\choice Reno, NV
		\choice San Antonio, TX
		\choice These cities' trends sounded too similar
		\choice I could not identify which city was which
		\end{choices}
		
	\begin{solution}
	Los Angeles and Indianapolis both have slopes of $\sim -4.1\%$ yr$^{-1}$. San Antonio and Reno are much lower, $\sim -1.7\%$ yr$^{-1}$.  If instead we just consider the average of the first and last years, LA changed $-49\%$, San Antonio $-39\%$, Indianapolis $-37\%$, and Reno $-15\%$.
	\end{solution}		
		
	\part Rate your confidence in this answer.
	\fiveptgna{Minimal confidence}{Extremely confident}{Could not tell}
	
	\part Now turn the cities on and off as you wish. Now which city would you say has the largest relative change in \ce{NO2}? \\
	\begin{choices}
	\CorrectChoice Indianapolis
	\CorrectChoice Los Angeles, CA
	\choice Reno, NV
	\choice San Antonio, TX
	\choice These cities' trends sounded too similar
	\choice I could not identify which city was which
	\end{choices}
	
	\part Rate your confidence in this answer.
	\fiveptgna{Minimal confidence}{Extremely confident}{Could not tell}
	
\instructions{Change the scaling to \textbf{ABSOLUTE} and make sure all four cities (Indianapolis, Los Angeles, Reno, and San Antonio) are on.}	

	\part Keep all four cities on. Which city has the largest absolute change in \ce{NO2}? \\
		\begin{choices}
		\choice Indianapolis
		\CorrectChoice Los Angeles, CA
		\choice Reno, NV
		\choice San Antonio, TX
		\choice These cities' trends sounded too similar
		\choice I could not identify which city was which
		\end{choices}
		
	\begin{solution}
	Los Angeles has a slope of $\sim -7.3 \times 10^{14}$ molec. cm$^{-2}$ yr$^{-1}$. Indianapolis is the next largest at $\sim -2.6 \times 10^{14}$ molec. cm$^{-2}$ yr$^{-1}$.  The differences between the average of the first and last year are $-7.3 \times 10^{14}$, $-2.7 \times 10^{14}$, $-0.9 \times 10^{14}$, and $-0.8 \times 10^{14}$ molec. cm$^{-2}$ yr$^{-1}$ for LA, Indianapolis, Reno, and San Antonio, respectively.
	\end{solution}		
		
	\part Rate your confidence in this answer.
	\fiveptgna{Minimal confidence}{Extremely confident}{Could not tell}
	
	\part Now turn the cities on and off as you wish. Now which city would you say has the largest absolute change in \ce{NO2}? \\
	\begin{choices}
	\CorrectChoice Indianapolis
	\CorrectChoice Los Angeles, CA
	\choice Reno, NV
	\choice San Antonio, TX
	\choice These cities' trends sounded too similar
	\choice I could not identify which city was which
	\end{choices}
	
	\part Rate your confidence in this answer.
	\fiveptgna{Minimal confidence}{Extremely confident}{Could not tell}
	\end{parts}
	
	
\instructions{\textbf{PAUSE} the playback. Turn \textbf{ON} all sites. Turn \textbf{OFF} the site-type specific sounds. Jump back to the beginning of the record and just stay on 2005-01-01.}

\question Regional trends:
	\begin{parts}
	\part Which region had the greatest \ce{NO2} in 2005? (Users testing in stereo may circle both east or both west regions if you cannot distinguish north and south.)\\
		\begin{choices}
		\CorrectChoice Northeast
		\choice Southeast
		\choice Northwest
		\choice Southwest
		\choice These regions sounded too similar
		\choice I could not identify which region was which
		\end{choices}
		
	\begin{solution}
	I don't have specific numbers for this one at the moment, but the northeast (esp. NY-Philly-NJ-Baltimore-DC) had so much \ce{NO2} their plumes were essentially indistinguishable from space. Los Angeles' signal may be large enough to make the SW a reasonable alternative sonically.
	\end{solution}		
		
	\part Rate your confidence in this answer.
	\fiveptgna{Minimal confidence}{Extremely confident}{Could not tell}
	\end{parts}
	
\instructions{For these next questions, start with only the rural sites \textbf{ON}. Turn on and off the sites and adjust the speed as needed to explore the data and answer the questions. Set scaling to \textbf{RELATIVE TO CATEGORY MAXIMUM}. Turn site-type-specific sounds back on.}

\question Trends at rural sites:
	\begin{parts}
	\part Identify the two rural sites with the greatest change in \ce{NO2}.
	\begin{solutionorbox}[1cm]
	Shenandoah and Rural AL
	\end{solutionorbox}
	
	\part Rate your confidence in this answer.
	\fiveptgna{Minimal confidence}{Extremely confident}{Could not tell}
	\end{parts}
	
\instructions{Now, turn \textbf{OFF} all sites \emph{except} Phoenix, AZ. Change panning to \textbf{\ce{NO2}/\ce{O3}}. Change scaling to \textbf{RELATIVE TO INDIVIDUAL MAXIMUM}. Turn \ce{O3} \textbf{ON}. Adjust the speed as needed to answer the question.}

\question \ce{NO2}/\ce{O3} relationship:
	\begin{parts}
	\part Satellite measurements of surface \ce{O3} are highly uncertain, so the data here may not represent the relationship between \ce{NO2} and \ce{O3} accurately. Nevertheless, based on the sonification, would you say \ce{NO2} and \ce{O3} concentrations are correlated, anti-correlated, or have no significant relationship?
	\fiveptcorrna
	
	\begin{solution}
	Technically, the correlation of $\Delta \ce{O3}$ and $\Delta \ce{NO2}$ is negative with a $p$ value of 0. However, the $R^2$ is $< 0.1$, so either ``weakly anti-correlated'' or ``no significant correlation'' are acceptable.
	\end{solution}
	
	\part Rate your confidence in this answer.
	\fiveptgna{Minimal confidence}{Extremely confident}{Could not tell}
	\end{parts}

\end{questions}

\section*{Unguided exploration and interface feedback}
\begin{questions}


\question Effectiveness of panning:
	\begin{parts}
	\part How effectively did the panning (distribution of sound between the left and right/8 surround channels) communicate the position of various sites?
	\fiveptgeneric{Very ineffective}{Very effective}
	
	\part Additional comments about the panning:
	\makeemptybox{1in}
	
	\part If you listened on the 8-channel setup, please comment on the mapping of longitude in the forward and back direction in the panning. Was it intuitive that north = forward or not?
	\makeemptybox{1in}
	\end{parts}


\question 
	\begin{parts}
	\part If you had one of each type of site playing (city, power plant, and rural), how well could you distinguish each type by its sound?
\fiveptgeneric{Not at all}{Very well}	

	\part Additional comments about the different sounds for different types of sites:
	\makeemptybox{1in}
	\end{parts}


\question How engaging was the aural presentation?
\fiveptgeneric{Not at all}{Very engaging}

\question Any other observations about the trends of \ce{NO2}, \ce{O3}, or their relationship from your listening?
\makeemptybox{\stretch{1}}

\newpage
\question What elements of the \emph{sonification} (not the GUI) were most difficult to interpret or understand?
\makeemptybox{\stretch{1}}

\question What elements of the \emph{sonficiation} (not the GUI) were easiest to interpret or understand?
\makeemptybox{\stretch{1}}
\newpage

\newpage
\question What elements of the \emph{GUI} were most difficult to use or understand?
\makeemptybox{\stretch{1}}

\question What elements of the \emph{GUI} were easiest to use or understand?
\makeemptybox{\stretch{1}}
\newpage

\question Additional comments:
\makeemptybox{\stretch{1}}
\newpage

\end{questions}

\end{document}